{\setstretch{1}

\mypage
\subsection*{Teaching Quotes}

%\addcontentsline{toc}{section}{Teaching Quotes}

\myquote{I am not a teacher, but an awakener}{Robert Frost}
\myquote{I never teach my pupils. I only attempt to provide the conditions in which they can learn.}{Albert Einstein}
\myquote{What we learn with pleasure we never forget.}{Alfred Mercier}
\myquote{To teach is to learn twice over.}{Joseph Joubert}
\myquote{I am interested in helping students feel successful; to
attempt to remove the comfort zone of the passive learner and to
empower students to actively engage in their own learning as well as
that of their peers.}{Simon Albon}
\myquote{I believe passionately in David Suzuki's notion of
\enquote{the power of one,} that I can make a difference in my work
with prospective teachers and create a \enquote{ripple effect} in
education. This means I have a responsibility to be a role model as
a scholar, a teacher, and mentor to help my students implement
informed and thoughtful teaching practices, build communities of
inquirers in their classrooms and schools, and become lifetime
learners through ongoing professional development and study.}{Marilyn Chapman}

\mypage
\myquote{You can teach anyone who is interested in teaching to be
a good teacher: to set realistic objectives, to be well organized, to
be clear... What people who become truly exceptional teachers add to
these basic skills is they really truly care that their students
learn.}{Carol-Ann Courneya}
\myquote{An educated person is transformed by what she knows; it
changes her outlook, her horizons, and her habits of mind. To teach
someone is to participate consciously, and carefully, in that
transformation.}{Linda Farr Darling}
\myquote{I let students know that in real life, there's more than one
right answer. My job is to teach them how to approach a problem -- how
to learn, not just what to learn.}{Mary Ensom}
\myquote{I emphasize clarity and planning. Equally important are
a sense of humour and a courteous attitude.}{Keith Farquhar}
\myquote{The thing I enjoy most is teaching first-year courses.
A first-year course is not just an introduction to some subject -- it
is also an opportunity for the instructor to introduce students to his
or her passion for that subject.}{Christopher Friedrichs}

\mypage
\myquote{It is a real privilege to teach students -- an opportunity to
\enquote{turn them on} as opposed to doing the reverse. And the key to
\enquote{turning them on?} I feel it has to do with believing in what you are
selling. For if you don't and would rather be elsewhere, students pick
that up and you impact lessons. So, if there is one quote from me, it
would be: It's not hard to sell something (indeed some\textbf{one}) you believe
in.}{Dan Gardiner}

\myquote{Teaching students less can help them remember more.}{Lee Gass}
\myquote{Students seem to learn best when their curiosity is engaged
and they can feel the link between the material and their own past,
present or future lives. Given this, good teaching is about fostering
curiosity. It is about finding, fueling, and firing up the links and
creating experience within and between learners. Course content exists
to be played with, tossed around until it becomes pertinent.}{Clarissa Green}
\myquote{My basic teaching philosophy is to be enthusiastic about my
subject, honest about potential confusions, and unambiguous in my
explanations}{Chris Orvig}
\myquote{My philosophy of teaching is to provide an atmosphere of
caring and respect for students in order to promote free-thinking and
independent learning.}{Wayne Riggs}

\mypage
\myquote{My philosophy with regard to education is that the teacher
should facilitate the student in their search for knowledge (rather
than deliver the knowledge per se). With this in mind I try to engage
the students in an interactive discussion around the specific subject
matter which they need to know.}{Niamh Kelly}
\myquote{I believe teaching should promote the autonomy of students;
elicit the preoccupations, passions, and lived experience students
bring to the academy; and create a space for articulation and
scholarly inquiry of these elements in their research and in their
writing. My priority is creating a learning environment with
structures that engage student participation and whereby their
participation shapes the learning environment.}{Karen Meyer}
\myquote{I believe that teaching is a creative art in which
evidence-based knowledge is applied toward meeting the learning goals
of learners. I believe that effective teaching is often the spark that
ignites the imagination, possibility, and promise for learners,
including the teacher.}{Barbara Paterson}
\myquote{Listen, respect, and respond to students' points of view; be
excited about mutual engagement in construction of knowledge; love
learning and its complexity. Be curious about and committed to the
possibilities inherent in the intersection of scholarship, mentorship,
and pedagogy.}{Marion Porath}

\mypage
\myquote{If your students are highly intelligent, strongly motivated
and well behaved, it matters little how you teach as they will thrive
anyway. If they aren't, your knowledge, preparation and skill as
a teacher do matter.}{William Webber}
\myquote{UBC is a big place, but it need not be an impersonal one. The
instructor's job is to remember that a big class is not just a sea of
faces, but a group of individuals with different backgrounds,
interests and enthusiasms.}{Christopher Friedrichs}
\myquote{My goal is to present topics with excitement and enthusiasm,
drawing on both work and research experiences to make material
relevant. Once students realize the value of their marketable skills
it increases their desire to master those skills and concepts.}{Wayne Riggs}
\myquote{I teach that a discovery is not simply a eureka moment, but
instead a process where you capture a glimpse of nature exposed,
convince your peers of what you saw and demonstrate utility. Eureka on
its own is merely an observation.}{Stephen W. Scherer}

}
