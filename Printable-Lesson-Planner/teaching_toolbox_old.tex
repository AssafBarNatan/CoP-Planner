\subsection*{Timed Think}
\begin{center}
  \begin{tabular}{rp{4in}}
    \textbf{What it is:}& Pose a question, give the students a set time 
    to think about the answer silently, then ask the question again and 
    call on a student to answer it.\\
    \textbf{Possible variants:}&Timed group discussion, timed pair discussion, 
    untimed lecture pause (ie, pause lecture until enough hands are raised).\\
    \textbf{Uses:}& In the case of a silent class, this will generally force 
    an answer from the room. In feedback, students often want more time to 
    think about a problem after it is asked.\\
    \textbf{Notes:}& 
    \begin{minipage}{0.8\textwidth}
      \vspace{0.5cm}
      $\rule{\textwidth}{0.15mm}$\\
      $\rule{\textwidth}{0.15mm}$\\
      $\rule{\textwidth}{0.15mm}$\\
      $\rule{\textwidth}{0.15mm}$\\
      $\rule{\textwidth}{0.15mm}$\\
      $\rule{\textwidth}{0.15mm}$\\
      $\rule{\textwidth}{0.15mm}$\\
      $\rule{\textwidth}{0.15mm}$
    \end{minipage}
    \\
  \end{tabular}
\end{center}

\subsection*{Think-Pair-Share}

\begin{center}
  \begin{tabular}{rp{4in}}
    \textbf{What it is:}& Pose a question, give the students time to think about the solution to themselves quietly, then instruct students to pair up and discuss their solutions. Finally, ask students to share their answers with the class.\\
    \textbf{Possible variants:}&Think-Pair-Square (instead of sharing with the class, pairs share their answer with another pair\\
    \textbf{Notes:}& 
    \begin{minipage}{0.8\textwidth}
      \vspace{0.5cm}
      $\rule{\textwidth}{0.15mm}$\\
      $\rule{\textwidth}{0.15mm}$\\
      $\rule{\textwidth}{0.15mm}$\\
      $\rule{\textwidth}{0.15mm}$\\
      $\rule{\textwidth}{0.15mm}$\\
      $\rule{\textwidth}{0.15mm}$\\
      $\rule{\textwidth}{0.15mm}$\\
      $\rule{\textwidth}{0.15mm}$
    \end{minipage}
    \\
  \end{tabular}
\end{center}

\subsection*{Think-Aloud Pair Problem Solving}
In pairs, assign each student the role of listener or speaker. The speaker works through the problem out loud while the listener follows. You can do this with many problems, switching the roles of the students each time. 
\subsection*{Send a Problem}
Have students form groups and give each group a problem. Each group tries to solve their problem then passes the problem to the next group. The groups try to solve the new problem without looking at the previous solution. You can do as many rounds as you would like and the activity ends with students looking at all the answers to one problem and evaluating which one is the best. 
\subsection*{Punctuated Lecture}
(fast slides with questions that students work in groups to 
answer, things along the lines of: "why does this work?")
\subsection*{Free-For-All Online Discussion}
(TopHat does this well)
\subsection*{1-Minue Essay}
\subsection*{Key points/Important Take-aways}
Have students summarize the key points of the class or concept and collect the answers. 
\subsection*{Muddiest point/Clearest Point}
Ask students what the muddiest point of the lecture was for them and what was the clearest point and collect the answers. 
\subsection*{Ticket-out-the-door}
Give students a ``ticket'' with a prompt that gets them to reflect on the class (for example, this could be combined with a 3-2-1). Collect the tickets as students go out the door. 
\subsection*{Fill-in Blanks}
\subsection*{Write \& Quiz}
(students come up with a question, then quiz their partner)
\subsection*{Concept Map}
\subsection*{History Lesson}
(see: Leibniz stuff from last semester)
\subsection*{Sorting and Going Through the Steps}
(sort the steps (or write them down) then go through 
the steps one by one. This is good with a handout)
\subsection*{Critical Incident Questionnaire}
\subsection*{The Tommy Question}
(A.K.A The False Explanation. A fake student (Tommy) writes something,
makes a common misconception, and the students work in groups 
to fix it.)
\subsection*{Playing Darts}
(tun your back to class at the start, and students yell concepts 
related to the material. You write down. At the end, check off what was 
discussed.)
\subsection*{Ice Cream Sandwich}
(AKA: neapolitan ice-cream)
(students write:
1. Something they've mastered in the unit
2. Something they're struggling with
3. Something that was cleared up)
\subsection*{3-2-1}
Ask students to write down 3 things they learned this class, 2 things they found interesting and 1 question they still have.
\subsection*{Paper Passaround}
(I post three questions on the board, 
and groups can choose which to answer. They write answer, and pass 
around to other groups who critique it.)
\subsection*{Write the Test}
(this works well with a TopHat discussion or a public document: 
students spend five minutes coming up with possible test questions, 
and then have a resource to study from)
\subsection*{Grade A Question}
(give a screenshot of a bad solution, students grade it. 
see: Tommy question)
\subsubsection{Snowball Activity}

\begin{center}
  \begin{tabular}{rp{4in}}
    \textbf{What it is:}& Give students a prompt. Have students write their answer on a piece of paper, then crumple it up and throw the paper across the room. Each student picks up a snowball and adds a comment/question. The students them throw the ball again. Each student picks up a new snowball and now you can have a class discussion prompted by what was written on the snowballs. You can also collect the snowballs at the end of class. \\
    \textbf{Possible variants:}&\\
    \textbf{Notes:}& 
    \begin{minipage}{0.8\textwidth}
      \vspace{0.5cm}
      $\rule{\textwidth}{0.15mm}$\\
      $\rule{\textwidth}{0.15mm}$\\
      $\rule{\textwidth}{0.15mm}$\\
      $\rule{\textwidth}{0.15mm}$\\
      $\rule{\textwidth}{0.15mm}$\\
      $\rule{\textwidth}{0.15mm}$\\
      $\rule{\textwidth}{0.15mm}$\\
      $\rule{\textwidth}{0.15mm}$
    \end{minipage}
    \\
  \end{tabular}
\end{center}
