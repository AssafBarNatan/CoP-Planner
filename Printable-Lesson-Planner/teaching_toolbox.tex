{\setstretch{1}

%% \addtoolboxitem{Title}{What is it?}{Variants}{Uses}

\addtoolboxitem{Timed Think}
{Pose a question, give the students a set time 
to think about the answer silently, then ask the question again and 
call on a student to answer it.}
{Timed group discussion, timed pair
discussion, untimed lecture pause (ie, pause lecture until enough
hands are raised).}
{In the case of a silent class, this will generally
force an answer from the room. In feedback, students often want
more time to think about a problem after it is asked.}

\addtoolboxitem{Think-Pair-Share (TPS)}
{With a question posed, students 
think silently about an answer. Then they pair up, and share their 
response with their neighbour.}
{Explain instead of share, 
Think-Group-Share, Think-Pair-Share 
with a classroom voting system (Think-Vote-Share-Vote), 
repeated Think-Pair-Share-Think-Pair-Share, Think-Pair-Share where during the "Pair" step the student must find a classmate who has a different answer to them.}
{This is the crux of active learning, 
and should be used for simple, conceptual questions, not 
long computational question.}

\addtoolboxitem{Punctuated Lecture}
{A fast-paced lecture, where at the end of 
every slide there is a comprehension-style question, such as 
\enquote{why does this computation work?}, \enquote{explain 
this step}. The question is then worked through using 
\textbf{TPS}.}
{Questions can be posed in an interactive voting system, see also:
\textbf{Going Through the Steps}}
{This is a good technique for getting 
through material that is emphasized in class, as opposed 
to reviewing pre-class material. It combos well with 
timed thinks and TPS.}

\addtoolboxitem{Free-For-All Online Discussion}
{A class-wide open discussion, where 
any student can contribute text or pictures to a forum-style 
tool. This can be done using TopHat, Google Docs, or Quercus.}
{Timed responses, group submission, 
see also: \textbf{Write \& Quiz}, \textbf{Write the Test}, 
and \textbf{1-Minute Essay}.}
{This can be used for getting a lot of ideas 
fast, and to consolidate student solutions for everyone 
to see and use later in studying.}

\addtoolboxitem{1-Minute Essay}
{Students write a 1-minute essay 
linking concepts, explaining a concept, or summarizing 
a concept learned in class. This can be done with 
or without a specific prompt.}
{5-minute essay, 1-minute paragraph, 
writing in groups, 1-minute list, end-of-class summary. See 
also: \textbf{ice cream sandwich}, \textbf{write \& quiz}, 
and \textbf{free-for-all discussion}.}
{This activity is a good conclusion to a 
lecture, module, or other topic. Walking around to pick 
students to read their sentences out loud to their 
neighbours can also be used to build comfort 
within a group.}

\addtoolboxitem{Fill-In Blanks}
{A short, fill-in-the-blanks 
pop-quiz, either on the board, on a slide, or through 
an interactive classroom response system.}
{Giving the word possibilities, see also: \textbf{1-minute essay}}
{This is a good activity to start a 
class or a topic, and to make sure that everyone 
has done the reading, is on the same page, 
and is ready to start learning.}

\addtoolboxitem{Write \& Quiz}
{Students come up with a question, 
then partner up and quiz each other.}
{Larger groups sharing the questions, forcing questions to be
conceptual or computational. See also: \textbf{free-for-all
discussion}, \textbf{write the test}, \textbf{paper passaround}.}
{This is a good activity when there is a lot 
of relatively straightforward material that would take a long 
time to go over, but should be spot-checked. This activity 
also gets students to think about what questions could appear 
on tests.}

\addtoolboxitem{Concept Map}
{A concept map is a directed 
or undirected graph with concepts for nodes and 
edges for connections between them. Edges should be labeled. 
The activity is to make a concept map.}
{Students can get: list of concepts, 
the concept map without edges, or the concept map 
with unlabelled edges.}
{This is an excellent review-session 
tool, as it takes up the entire class, and can cover a 
lot of material. The main goal of the activity is to 
introduce students to the idea, not to finish the map.}

\addtoolboxitem{Critique History \& News}
{A short lesson on the origins 
of the course content, or a news clipping related 
to it. The more primary sources, the better.}
{Having students see the 
primary source and critique it.}
{Some students love this, some hate 
it, but it can be used to show to students that math 
was always difficult, and that other people also 
make mistakes.}

\addtoolboxitem{Going Through the Steps}
{Write a sequence of steps to solve 
a problem, and go through them one by one using other 
teaching-toolbox tools.}
{Handout with steps, allow 
students to find the correct order of the steps. 
See also: \textbf{punctuated lecture}, 
\textbf{think-pair-share}.}
{This is a good activity to use 
to teach students a specific problem-solving strategy.}

\addtoolboxitem{The Tommy Question}
{Tommy writes an incorrect or incomplete solution to a problem, 
and the students need to fix the solution.}
{Use previous exam solutions, have students grade the response, 
let students address their explanation to Tommy}
{This is a good activity to target subtle misconceptions and 
to highlight common pitfalls students may encounter}

\addtoolboxitem{Round Robin}{Students get into groups and take turns highlighting key points from the lecture or course content. Possible variant: \textbf{Playing Darts} - Students shout concepts related to the material, and the 
instructor writes them down on the board. Finish with a follow-up at the end of class outlining what was and wasn't covered in class. 

Pairs well with a \textbf{timed think}}
{This activity can be an opener and a closer, and reminds 
students that they are also responsible for things not covered 
in class.}

\addtoolboxitem{Ice Cream Sandwich}
%(AKA: neapolitan ice-cream)
{Students write three things: something they've 
mastered in the unit, something they're struggling with, and something that was cleared up}{Any triple of questions can work!}{Student reflection, seeing learning as a dynamic process.}

\addtoolboxitem{Draw the Definition/Theorem}{Provide the students with a definition or theorem and have them illustrate this definition or theorem. Their drawing could be of an explicit example which works, or something more general.}{Provide the students with a definition or theorem and a sketch. Have students fill in anything which is missing, and have them colour-code parts of the definition or theorem and colour the corresponding part of the picture that colour as well.}{Allows students to engage with a definition or theorem and build some intuition about the concept.}

\addtoolboxitem{Geogebra Applets}{Geogebra has many free applets available on its platform. It is possible to find many applets which are interactive and allow students to interact with different concepts. Be sure to test out the applet before use to ensure it is what you're looking for!}{There are a variety of different applets available. Some of the applets can be an entire activity on their own, and others can be a supplement to help you illustrate an idea to students.}{Allows students to build intuition behind different concepts in a visual, yet interactive way.}

\addtoolboxitem{Jigsaw}{Break the class into groups and have each group solve a different part of a problem. At the end, every one comes together to synthesize their solutions to solve the main problem.}{This can be used to fill in tables, to see patterns emerging through examples, or to solve a larger problem as a group.}{This can be used to explore theorems or definitions and build intuition.}

\addtoolboxitem{Gameshow}{For review, have large a variety of questions prepared which you will display one at a time. Students will answer questions, and will keep track of their longest streak of correct answers.}{Could have the class break up into teams to play, or change the format to mimic a real game show (for example, Jeopardy).}{A fun way to engage students during a review class before a midterm or final exam.}

\addtoolboxitem{Paper Passaround}{In groups, students write things on papers, and exchange with other groups. The other group reads, and responds on the page.}{Some examples: each group solves one of three problems on the board, and then critiques another group's solution. See also: \textbf{Write \& Quiz}}{This is a good physical writing exercise that 
lets students see and critique how math can be communicated in writing.}

\addtoolboxitem{Run the Test}{Students play the role of the instructor in writing, grading, or helping students during a test-environment. For example, students may be given a solved test to critique, or could be asked to give advice to students before, after, or during the test (see: \textbf{Tommy question})}{An online tool to collect student-made test questions to create a question bank can be useful.}{This exercise can be used to demystify the test, and to highlight common mistakes. See also: \textbf{Write \& Quiz}, \textbf{Free-For-All Online Discussion}}


\mute{
-round robin: four key points from the class [Hannah]
-every group does a different computation/calculation/activity, and then 
it's synthesized together to make a rule/theorem/idea [Yvon]
- Think-Pair-share, then find someone who thinks otherwise, TPS, repeat. [added to think-pair-share]
- Geogebra applet [Yvon - Hannah]
- Gameshow: longest correct answer streak on TopHat. Ask a question, 
somehow track who got a correct answer streak. Students have a goal. [Yvon]
- Split the class into chunks, chunks answer together, and each chunk is 
competing.

\mypage
\subsection*{Ice Cream Sandwich}
(AKA: neapolitan ice-cream)
(students write:
1. Something they've mastered in the unit
2. Something they're struggling with
3. Something that was cleared up)
[Hannah]

\mypage
\subsection*{Paper Passaround}
(I post three questions on the board, 
and groups can choose which to answer. They write answer, and pass 
around to other groups who critique it.) [Hannah]

\mypage
\subsection*{Write the Test}
(this works well with a TopHat discussion or a public document: 
students spend five minutes coming up with possible test questions, 
and then have a resource to study from)

\mypage
\subsection*{Grade A Question}
(give a screenshot of a bad solution, students grade it. 
see: Tommy question)

\mypage
\subsection*{Pre-Class Reflection}

}
}

%% Adds some empty teaching toolbox to fill in your own things

\addblanktoolboxitem
\addblanktoolboxitem
\addblanktoolboxitem
\addblanktoolboxitem
\addblanktoolboxitem
